% trascrizione: Candido

% lo scrivo in formato lezione, andrà fatto il merge nella stesura finale
% forse sono stupido io, ma bisognerebbe avere delle reference a lemmi, proposizioni e teoremi

\begin{oss}
Sia $X$ l'insieme ambiente e $Y\subseteq X$, $x\in Y$.\\
Si ha che $x$ è interno a $Y \centernot\implies x$ è di acc. per $Y$, ad esempio per $(X, \tau _{disc})$ preso $Y\is\left\{ {x} \right\}$ allora $x$ è interno a $Y$, poiché quest'ultimo è aperto, ma ${x}\setminus {x} = \varnothing$ quindi non è di acc.
\end{oss}

\begin{prop}
$X 1$-numerabile e compatto \implies $X$ è compatto per successioni
\end{prop}
\begin{proof}
Sia $(a_n)_{n\in\N}$ una successione a valori in $X$. Si hanno allora due casi:
\begin{itemize}
\item se assume un numero finito di valori ne assumerà uno di questi infinite volte, quindi posso estrarre una sottosuccessione costante;
\item se assume un numero infinito di valori distinti allora per il lemma dimostrato si ha che $\exists x\in X$ di accumulazione per $\left\{ {a_n} \right\}$.
\end{itemize}
Nel secondo caso la successione si costruisce per $1$-numerabilità, infatti, presa una base di intorni $\left\{ {U_n} \right\}$ di $x$ \wlg annidati, costruisco la sottosuccessione prendendo come $n_0$ il minimo per cui $a_{n_0}\in U_0$ e $n_1 > n_0$ e che sia il minimo $t.c. a_{n_1}\in U_1$, \dots . Questa sottosuccessione ovviamente converge a $x$ come da definizione.
\end{proof}

\begin{oss}
Se un punto è di accumulazione per un dato insieme in uno spazio $T_2$ si ha che ogni suo intorno del punto interseca l'insieme in infiniti punti, infatti se così non fosse, usando il fatto che i punti sono chiusi in uno spazio $T_2$, potrei avere un intorno con intersezione finita. Posso sottrarre questi punti all'intorno, che continua a contenere il punto e ad essere aperto per intersezione finita, ma esso non interseca più l'insieme, assurdo
\end{oss}

\begin{prop}
$X 2$-numerabile e compatto per successioni $\implies$ X è compatto
\end{prop}
\begin{proof}
Sia dunque $R$ un ricoprimento di $X$, da esso ne voglio estrarre uno finito.\\
Si ottiene prima un risultato intermedio, cioè l'estrazione di un sottoricoprimento numerabile. In questo primo passaggio non è coinvolta l'ipotesi di compattezza per successioni.\\
Sia dunque $B$ una base numerabile di $X$. Si ha allora che $\forall x\in X \exists R_x \in R t.c x\in R_x$. Per definizione di base ho anche che $\exists B_x \in B t.c. B_x \subseteq R_x \et x\in B_x$. Da cui il primo risultato, poiché l'insieme dei $B_x$ è numerabile, essendo una sottofamiglia della famiglia numerabile $B$.\\
Il secondo parziale è estrarre da un ricoprimento numerabile un ricoprimento finito.\\
Per fare ciò si procede per assurdo: se $(R_n)_{n\in \N}$ non ammettesse un sottoricoprimento finito posso costruire la mia successione in modo che $a_0 \not \in R_0$ e $a_1 \not \in R_1 \cup R_2$, \dots . Risulta che da questa successione è impossibile estrarre una sottosuccessione convergente, infatti se convergesse a un punto dell'insieme starebbe definitivamente dentro l'aperto che lo contiene, da cui l'assurdo.
\end{proof}

\begin{cor}
In uno spazio $2$-numerabile si ha che: compatto \iff compatto per successioni
\end{cor}

\titlet{Compattezza in spazi metrizzabili}
Se uno spazio metrizzabile è compatto questa proprietà risulta avere conseguenze ``importanti'' su ogni spazio metrico che induce quella topologia.

\begin{defn}
Sia $(X,d)$ uno spazio metrico. Esso si dice \emph{totalmente limitato} se $\forall \varepsilon > 0 X$ è ricoperto da un numero finito di palle di raggio $\varepsilon$.
\end{defn}

\begin{prop}
Sia $(X,\tau _d)$ compatto per successioni, allora $(X,d)$, una metrica inducente, è totalmente limitato
\end{prop}
\begin{proof}
Per assurdo: $\exists \varepsilon > 0 t.c. X$ non è ricoperto da un numero finito di palle di raggio $\varepsilon$.\\
Allora dato $x_0 \in X$ si ha che $\exists x_1 \in X t.c. x_1 \not \in B_\varepsilon (x_0)$ e $\exists x_2 \in X t.c. x_2 \not \in B_\varepsilon (x_0) \cup B_\varepsilon (x_1)$, \dots .\\
Dunque per induzione costruisco una successione $x_n$ che non ammette nessuna sottosuccessione convergente, infatti i punti distano fra loro almeno $\varepsilon$, da cui l'assurdo.
\end{proof}

\begin{prop}
$(X, \tau _d)$ compatto per successioni \implies $X$ 2-numerabile
\end{prop}
\begin{proof}
$\forall n \in \N$ esiste una famiglia finita $\mathcal{F}_n$ di palle aperte di raggio $2^{-n}$ che ricopre $X$, per totale limitatezza.\\
Sia allora $\mathcal{B} \is \cup _{n \in \N} \mathcal{F_n}$. Si ha che $\mathcal{B}$ è sicuramente una famiglia numerabile, per verificare che è anche una base di $\tau _d$ si verificano i seguenti due punti:
\begin{itemize}
\item Se $\mathcal{A}$ è un ricoprimento di $X \implies \exists \bar{n} t.c.$ ogni palla di $\mathcal{F}_n$ è contenuta in uno degli aperti di $\mathcal{A}$;
\item Si applica il punto precedente $\forall A \in \tau _d , \forall x \in A$, considerando come ricoprimento $\mathcal{A} \is \left\{ {A, X\setminus \left\{ {x}\right\}}\right\}$
\end{itemize}
\end{proof}

\begin{cor}
$(X,\tau _d)$ metrizzabile \implies (compatto \iff compatto per successioni)
\end{cor}

\begin{prop}
$(\R^n, \tau _E), K\subset \R^n compatto \iff chiuso e limitato$
\end{prop}
\begin{proof}
Data l'equivalenza data dal corollario precedente si dimostrano le due implicazioni nel caso di compatto per successioni:
\begin{itemize}
\item \implies, per assurdo posso supporre che l'insieme sia illimitato, ma prendendo una sottosuccessione che tende a \infty si ha che nessuna sottosuccessione estratta può convergere, perché anch'essa tende a \infty, mentre negando la chiusura posso costruire una successione che tenda a un punto di accumulazione che non appartiene all'insieme, e per stabilità del limite nel passaggio a sottosuccessioni si ha che nessuna sottosuccessione può convergere a un punto dell'insieme;
\item \impliedby, si ottiene eseguendo estrazioni successive mediante il teorema di Bolzano-Weierstrass (compattezza per i chiusi e limitati di $\R$), il che è reso possibile dalla stabilità del limite nel passaggio a sottosuccessioni e dalla dimensione finita di $\R^n$
\end{itemize}
\end{proof}

\begin{teo}
Sia $(X,\tau _d)$ compatto per successioni, e $(X,d)$ una metrica inducente, allora ($(X,\tau _d)$ compatto \iff $(X,d)$ è totalmente limitato e completo)
\end{teo}
\begin{proof}
La dimostrazione è lasciata per esercizio al lettore, ma non è richiesta nel corso
\end{proof}
