% trascrizione: Petrillo

Adesso facciamo un teorema che intuitivamente vuol dire:
preso un diffeomorfismo di $\R^n$ in sé stesso che preserva l'orientazione,
possiamo deformarlo in modo liscio (``senza strappi'')
fino a farlo diventare l'identità.
Ma prima, un lemma utilissimo:

\begin{lemma}[Taylor dei geometri]
	\label{th:lemprec}
	Sia $\fundef{U\subseteq\R^n}\R^k$ liscia con $U$ aperto convesso, $0\in U$, $f(0)=0$.
	Allora esistono $n$ funzioni lisce $\fundef[g_i]{\R^n}\R^k$ tali che $f(x)=\sum_ix_ig_i(x)$.
\end{lemma}

\begin{proof}
	Lavorando componente per componente, ci restringiamo al caso $k=1$.
	Per il teorema fondamentale del calcolo abbiamo:
	\[f(x)=
	f(1\cdot x)-f(0)=
	\int_0^1\deriv{}t\big(f(tx)\big)\de t=
	\sum_{i=1}^n\left(\int_0^1\pderiv f{x_i}(tx)\de t\right)x_i\]
	Allora poniamo:
	\[g_i(x)\is\int_0^1\pderiv f{x_i}(tx)\de t
	\qedhere\]
\end{proof}

\begin{teo}[Linearizzazione dei diffeomorfismi di $\R^n$ a meno di isotopia]
	Gli autodiffeomorfismi euclidei che preservano l'orientazione sono diff-isotopi all'identità:
	\begin{equation*}
		\begin{rcases}
			\fundef{\R^n}{\R^n}\text{ diffeomorfismo} \\
			\forall x\in\R^n:\det(\de_xf)>0
		\end{rcases}
		\implies \exists F:
		\begin{cases}
			\fundef[F]{\R^n\times[0;1]}{[0;1]}\text{ liscia} \\
			\forall t\in[0;1]:F_t\text{ diffeomorfismo} \\
			\hspace{-1ex}\begin{array}{l}
				F_0=f \\
				F_1=\id
			\end{array}\quad\text{dove }F_t\is F|_{\R^n\times\set t}
		\end{cases}
	\end{equation*}
\end{teo}

\begin{proof}
	Non costruiamo subito la $F$:
	esibiremo vari cammini del tipo di $F$ ma che non collegano direttamente $f$ a $\id$;
	la loro concatenazione darà $F$.
	
	\newcommand*\ISOMOUSE[2]{\fbox{${#1}\longrightarrow{#2}$}}
	\begin{description}
		\item[\ISOMOUSE{f}{f(0)=0}]
			Sia $x_0\is f(0)$.
			Allora $H(x,t)\is f(x)-tx_0$ collega $f=H_0$ a $f'\is H_1$ con $f'(0)=0$,
			e $H_t$ è sempre un diffeomorfismo.
			Per leggerezza di notazione diciamo che $f(0)=0$.
		\item[\ISOMOUSE{f}{\de_0f}]
			Per definizione, $\de_0f(x)=\lim_{t\to 0}\frac{f(tx)}t$.
			Allora poniamo:
			\[\tilde F(x,t)\is
			\begin{dcases}
				\frac{f(tx)}t & t\in[0;1] \\
				\de_0f(x) & t=0
			\end{dcases}\]
			Verifichiamo che $\tilde F$ è liscia.
			Applicando il \autoref{th:lemprec} a $f$ abbiamo $\tilde F(x,t)=\sum_ix_ig_i(tx)$ per $t\neq 0$.
			Ma, osservando la forma esplicita delle $g_i$,
			vediamo che $g_i(0)=\pderiv f{x_i}(0)$,
			quindi l'uguaglianza si estende a $t=0$.
		\item[\ISOMOUSE{\de_0f}{B_1\in\SO_n}]
			Sia $A$ la matrice di $\de_0f$,
			per ipotesi $\det A>0$.
			Allora esiste un unico prodotto scalare definito positivo $\scp{}{}_A$
			tale che le colonne $A^i$ sono una base ortonormale;
			sia invece $\scp{}{}_I$ quello canonico.
			Osserviamo che
			\[\scp{}{}_t\is (1-t)\scp{}{}_A+t\scp{}{}_I\qquad t\in[0;1]\]
			ci dà un cammino di prodotti scalari definiti positivi.
			Per ogni $\scp{}{}_t$ applico Gram-Schmidt alle $A^i$ ottenendo una base ortonormale
			$\set{B_t^1,\dots,B_t^n}$
			dove consideriamo le $B^i_t$ come colonne di una matrice $B_t$.
			Basta dunque osservare che:
			\begin{itemize}
				\item le formule di Gram-Schmidt per le $B^i_t$ sono lisce;
				\item $\det B_t>0$;
				\item $B_0=A$;
				\item $B_1\in\SO_n$.
			\end{itemize}
		\item[\ISOMOUSE{B_1}{\id}]
			\marginpar{Non sono riuscito a formalizzare per bene il procedimento,
			che in effetti è geometricamente intuitivo:
			ruoto lungo un asse alla volta.
			Benedetti ha datto due hint:\\
			\emph{A lezione}: considero che le colonne $B_1^i$ sono una base ortonormale
			rispetto al prodotto canonico, ruoto e cambio base;\\
			\emph{Sulle dispense}: induzione su $n$ partendo da $\SO_2$.}
			È geometricamente intuitivo collegare una matrice di $\SO_n$ all'identità.
	\end{description}
	
	Per concatenare questi cammini in modo liscio,
	basta prima comporre la variabile $t$ con una funzione liscia $\fundef[\varphi]\R\R$ tale che:
	\[\varphi(0)=0, \quad
	\varphi(1)=1, \quad
	\forall k\in\N:\frac{\de^k\varphi}{\de t^k}(0)=\frac{\de^k\varphi}{\de t^k}(1)=0.\]
	Un esempio di tale funzione è:
	\[\varphi(x)=\frac{\phi_{0,\frac32}(x)}{\phi_{0,\frac32}(x)+\phi_{-\frac12,1}(x)},\quad
	\phi_{a,b}(x)=\begin{cases}
		e^{\frac1{(x-a)(x-b)}} & x\in(a;b) \\
		0 & \text{altrimenti}
	\end{cases}\]
\end{proof}

Sia $\fundef U{\R^n}$ con $f$ liscia, $0\in U\subseteq\R^n$, $U$ aperto, $f(0)=0$.

Espandiamo in serie la $f$:
\[f(x)=
\de f_{0}(x)+\transp{x}(Hf_{0})x+R
\quad\text{con}\quad
\lim_{x\to0}\frac R{\norm{x}^2}=0\]
Dove $Hf$ è la matrice hessiana:
\[(Hf_{0})_{ij}\is
\frac{\partial^2f}{\partial x_j\partial x_i}(0)\]
Ricordiamo che l'hessiana è simmetrica per le funzioni lisce.

Se $\de f_{0}\neq0$ cioè $\nabla f(0)\neq0$, siamo nelle ipotesi del \autoref{th:funimpsurg}. Allora $\exists$~parametrizzazione locale $\psi$ tale che $f\circ\psi(x)=x_1$.

Supponiamo che $\de f_{0}=0$. Allora chiamiamo $0$ \emph{punto critico} di $f$. In questo caso $f(x)=\transp{x}(Hf_{0})x+R$.
\marginpar{scrivere definizione per bene}

Supponiamo che sia un punto critico \emph{non degenere}, cioè $\det(Hf_{0})\neq0$. Allora applichiamo il teorema di Sylvester alla matrice $Hf$ (che è simmetrica e non degenere), avremo indici di positività e negatività $i_+$ e $i_-=n-i_+$ e:
\[\exists P\in\GL_n:
\transp P(Hf_{0})P=
\begin{pmatrix}
	\begin{matrix}
		1 & & \\
		& \ddots & \\
		& & 1 \\
	\end{matrix} & \Bigg\}i_+ \\
	i_-\Bigg\{ & \begin{matrix}
		-1 & & \\
		& \ddots & \\
		& & -1 \\
	\end{matrix}
\end{pmatrix}\si J\]

Definiamo le forme quadratiche: $Q_H(x)\is\transp{x}(Hf_{0})x$, $Q_J(x)=x_1^2+\dots+x_{i_+}^2-(x_{i_++1}^2+\dots+x_n^2)$. Allora $f(x)=Q_H(x)+R$.

Chiamiamo $i_-$ l'indice del punto critico non degenere.
\marginpar{scrivere definizione per bene}

\begin{ex}
	La proprietà di punto critico e il suo indice sono invarianti per riparametrizzazione differenziabile a sinistra.
\end{ex}

\begin{teo}[Lemma di Morse]
	Sia $f$ come sopra ma a valori in $\R$, allora esiste una riparametrizzazione locale intorno all'origine che trasforma la $f$ nella forma quadratica canonica del tipo di $Q_J$:
	\[\begin{rcases}
		f\in C^\infty(U,\R^n) \\
		\text{$U$ intorno di $0$ in $\R^n$} \\
		\text{$0$ punto critico di indice $\lambda$} \\
		f(0)=0
	\end{rcases}\implies
	\exists\psi,U',U'':\begin{dcases}
		U''\subseteq\R^n \\
		U'\subseteq U \\
		\text{$U'$ intorno di $0$} \\
		\fundef[\psi]{U''}{U'} \\
		f\circ\psi(x)=-\sum_{i=1}^\lambda x_i^2+\sum_{i=\lambda+1}^nx_i^2
	\end{dcases}\]
\end{teo}

\begin{proof}
	\wlg restringiamoci a $U$ convesso.
	
	Applichiamo due volte il \autoref{th:lemprec}:
	\begin{align*}
		f(0)=0 &\so f(x)=\sum_ix_ig_i(x) \\
		\de f_{0}\equiv0 &\so f(x)=\sum_{ij}b_{ij}(x)x_ix_j
	\end{align*}
	Dove le $b_{ij}$ sono lisce. \wlg $b_{ij}=b_{ji}$ perché simmetrizzo con $b_{ij}\mapsto\frac{b_{ij}+b_{ji}}2$, che a meno di restringere $U$ non cambia l'indice di $(b_{ij})\si B$ che in effetti è $\lambda$ (a meno di un cambiamento lineare di base, posso suppore $Hf_{0}=J$).
	\marginpar{spiegare per bene}
	
	Applichiamo Gram-Schmidt alle $B$:
	\[\forall x\,\exists P(x)\in\GL_n:
	\transp PBP=
	\begin{pmatrix}
		\begin{matrix}
			-1 & & \\
			& \ddots & \\
			& & -1 \\
		\end{matrix} & \Bigg\}\lambda \\
		n-\lambda\Bigg\{ & \begin{matrix}
			1 & & \\
			& \ddots & \\
			& & 1 \\
		\end{matrix}
	\end{pmatrix}\si J\]
	Osserviamo che $P(0)=I$ e che le $P$ sono lisce.
	
	Sia $\fundef[\varphi]U{\R^n}$ con $\varphi(x)\is P(x)^{-1}$. Osserviamo che $\de\varphi_{0}=\id$.
	
	Applichiamo il teorema della funzione inversa: $\varphi$ è un diffeomorfismo se restringiamo il codominio all'immagine.
	
	Abbiamo finalmente $f=\transp\varphi(\transp PBP)\varphi=\transp\varphi J\varphi$.
\end{proof}
