% Valerio

%anche come fosse antani scappella per tre

\newcommand*\Ps{\mathbb{P}} % Projective space
\newcommand*\Ro[1][n]{\mathbb{R}_0^{#1}} % set of real number vectors excluding origin
\newcommand*\tc{\ \text{t.c.} \ } % tale che

\titlet{Spazio proiettivo (reale)}
Sia $\Ro \is \R^n \setminus \set{0}$ (come sottospazio topologico); introduciamo su $\Ro[n+1]$ la relazione $\sim $ (di equivalenza proiettiva): \[
x, y \in \Ro[n+1], \ x \sim y \means  \exists \lambda \! \in \! \R \tc x = \lambda y \]

Le relative classi di equivalenza sono date da $[x]_\sim = \Span(x) \setminus \set{0}$.

Inoltre, essendo $S^n \subset \Ro[n+1]$, posso restringere tale relazione di equivalenza a $S^n$, ottenendo $\sim_S$ (rispetto a cui le classi di equivalenza sono date da $[x]_{\sim_S} = \set{x, -x} = [x]_\sim \, \cap \, S^n$).
Consideriamo ora gli insiemi quoziente $X \is \quoset{\Ro[n+1]}{\sim}$ e $Y \is \quoset{S^n}{\sim_S}$, con le rispettive proiezioni $\fundef[\pi]{\Ro[n+1]}{X}$ e $\fundef[\pi_S]{S^n}{Y}$.

\begin{oss}
	L'applicazione $[x]_\sim \! \mapsto \! \Span(x)$ è una biiezione tra $X$ e \break
	$\setdef[V \subset \R^{n+1}]{V \text{ è SSV\footnotemark{} di dimensione 1}}$.
	\footnotetext{Sottospazio vettoriale}
\end{oss}

\begin{prop}
	La biiezione $[x]_{\sim_S} \! \mapsto \! [x]_\sim$ (con inversa $[y]_\sim \! \mapsto \! [y]_\sim \cap S$) è un omeomorfismo tra $X$ e $Y$ (rispetto alle topologie quoziente).
\end{prop}

Si ricorda che gli aperti della topologia quoziente sono le immagini degli aperti saturi rispetto alla proiezione.

\begin{defn}[Cono]
	Un sottoinsieme $C\subseteq\Ro$ è un \emph{cono} se $\Span(C)\!\setminus\!\set{0}\subseteq C$ (e dunque, poiché $x\!\in\!\Span(x)$, se $\Span(C)\!\setminus\!\set{0}=C$).
\end{defn}
\begin{lemma}
	 Valgono le seguenti:
	\begin{itemize}
		\item $A \subseteq \Ro[n+1]$ è $\pi$-saturo $\leftrightarrow$ $A$ è un cono.
		\item $A \subseteq S^n$ è $\pi_S$-saturo $\leftrightarrow \, \forall x \!\in\! A \ -\!x \!\in\! A$.
	\end{itemize}
\end{lemma}

Pertanto, la biiezione tra $\setdef[A \subseteq {\Ro[n+1]} ]{A \text{ è $\pi$-saturo}}$ e $\setdef[A_S\subseteq S^n]{A_S \text{ è $\pi_S$-saturo}}$ data semplicemente da $A\mapsto A\cap S^n$, $\Span(A_S)\!\setminus\! \set{0} \mapsfrom A_S$ manda aperti in aperti e viceversa, mostrando continuità e apertura di $[x]_{\sim_S} \! \mapsto \! [x]_\sim$, che dunque è realmente un omeomorfismo; $X$ e $Y$ sono dunque topologicamente equivalenti.
