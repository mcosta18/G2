%Marco Costa

\begin{oss}
 $(g\circ h )_* = g_*\circ h*$
 \id_* = \id
\end{oss}
$g_*(\left[ X f\rightarrow Y_1 \right] = \left[X g\circ f \rightarrow Y_2 \right]$
\begin{oss}
 Se $\fundef[g]Y_1Y_2$ è diffeomorfismo, $\exists \fundef[g^{-1}Y_2Y_1]$ liscia $\impliesg_*$ è isomorfismo di spazi vettoriali.
\end{oss}
\begin{def}[Omotopia]
 $\fundef[g_0]Y_1Y_2$, $\fundef[g_1]Y_1Y_2$ sono \emph{omotope} se $\exists \fundef[G]{Y_1\times\left[ 0,1\right]}Y_2$ tale che,
 detta $G_t\is  G|_{Y_1\times \left \{ 1\right \}}, G_0=g_0, G_1=g_1$.
\end{def}
 \begin{oss}
  L'omotopia è un particolare cobordismo (basta considerare il cilindro associato!)
 \end{oss}
\begin{prop}
 $g_0, g_1$ omotope$\implies g_{0*}=g_{1*}$.
\end{prop}
\begin{proof}
 Inserire disegni
\end{proof}
\begin{def}[Cobordismo orientato]
 $X_0\sim_{cob^+}X_1$ se $\exists$ triade $(W, Z_0, Z_1)$ con $W$ orientata e isomorfismi $\fundef[\phi_0]X_0Z_0, \fundef[\phi_1]X_1Z_1$
 che preservano l'orientazione.
\end{def}
\begin{oss}
 Anche in questo caso possiamo aggiungere la struttura di gruppo abeliano, con la stessa operazione definita nella lezione precedente.
 In particolare $-\left[ X\right]=\left[-X\right]$. (Viene scambiata l'orientazione di tutte le componenti connesse)
\end{oss}
