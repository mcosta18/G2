%autore: Candido

\begin{oss}[Alcuni $\eta_n(Y)$ e $\Omega_n(Y)$]
~\\
Al fine di mostrare che gli spazi in esame non sono tutti banali si esaminano alcuni di questi in particolare.
Prima di tutto si esamina la struttura di varietà differenziabili in dimensione bassa (in particolare $n = 0, 1$)
\begin{description}
\item [n = 0] $X \in \{$0-varietà compatte connesse$\}$/$\{$diffeomorfismi$\}$ $\implies X = \set{pt.}$, mentre in generale, in mancanza di connessione, si avrà un'unione disgiunta di punti;

$X \in \{$0-varietà compatte orientate connesse$\}$/$\{$diffeomorfismi che preservano l'orientamento$\}$ $\implies X = \set{pt., \pm}$, dove il segno specifica l'orientazione, e nel caso generale non connesso si ha un'unione disgiunta di punti con segno;
\item [n = 1] $X \in \{$1-varietà compatte connesse$\}$/$\{$diffeomorfismi$\}$ $\implies X = [a,b]$ nel caso di varietà con bordo $~\vel~ X = S^1$ nel caso di mancanza di bordo, al solito nel caso generale si ha un'unione disgiunta degli spazi descritti (quest'ultimo fatto è vero, ma la dimostrazione è non banale);

$X \in \{$1-varietà compatte orientate$\}$/$\{$diffeomorfismi che preservano l'orientamento$\}$ allora $X$ è uno spazio di quelli descritti nel caso non orientato, dove l'orientazione è definita per ogni componente connessa scegliendo il verso orario o antiorario per ogni componente senza bordo (di tipo $S^1$) e positivo o negativo (con riferimento alla struttura ordinata di $\R$) per le componenti con bordo (di tipo intervallo chiuso).\\
Sulle componenti con bordo si ha inoltre un'orientazione indotta, conformemente alla convenzione stabilita \emph{"prima la normale uscente"}. Perciò nell'intervallo $[a,b]$ si ha che il punto $a$ ha segno $+$ se la normale uscente è concorde con l'orientamento dell'intervallo, cioè se $[a,b]$ è preso con verso negativo, $-$ altrimenti. Stessa cosa per $b$.
\end{description}

Si passa ora a $\eta_0(Y)$, con $Y$ varietà compatta chiusa (cioè senza bordo).
\end{oss}

\titlet{Trasversalità}

\epigraph{Se io vi dico che una cosa è vera, potete anche fregarvene del perché sia vero, insomma... potete anche fregarvene della dimostrazione!\\ In fondo siete fisici... siete abituati a digerire anche di più!}{R.Benedetti}

Si vuole introdurre un concetto secondo cui ci siano alcuni oggetti siano "speciali" rispetto ad altri. Seguono dunque alcuni esempi:

\begin{es}
\begin{itemize}
\item In $\R^2$ quali sono i sottospazi di $dim = 1$ rispetto ad una data retta? Solo se stesso, se come discriminante si considera lo spazio generato dalla retta fissata con l'altra arbitraria. Infatti tutti generano il piano di $dim = 2$, tranne la retta stessa, che con se stessa genera uno spazio di $dim = 1$ (sempre sé);
\item Per i piani nello spazio relativamente a una retta, o le rette relativamente ai piani, o sottospazi in dimensione più elevata si hanno delle estensioni naturali;
\item Per le curve in uno spazio di $dim \geq 2$ si ottiene una nozione analoga considerando i punti di intersezione e gli spazi tangenti in tali punti. Cioè sugli spazi tangenti si hanno le stesse relazioni descritte mediante l'algebra lineare.
%FIGURA con curve intersecanti e spazi tangenti
\end{itemize}
\end{es}

\begin{oss}
La relazione di essere "non speciale" è aperta, poiché per piccole deviazioni da un elemento generico si continuano ad avere elementi generici, inoltre gli elementi generici sono densi nell'insieme di tutti gli elementi.
\end{oss}

Ancora nelle precedente osservazione si stava trattando con concetti intuitivi, si da dunque ora una definizione formale del concetto di "speciale".

\begin{defn}[Trasversalità]
Sia $Y$ una varietà e $A\subseteq Y$ una sottovarietà, con $\partial Y = \partial A = \nullset$. Sia $X$ compatta e chiusa ($\partial X = \nullset$), e sia $\fundef[f]{X}{Y}$ liscia.

Si definisce allora $f$ \emph{trasversa} ad $A$ se: $\forall x \in X~$t.c.$~y = f(x) \in A~$ si ha che $~T_yY = T_yA + D_xf(T_xX)$.

Si scrive inoltre che $f \pitchfork A$.
\end{defn}

\begin{es}
$Y = \R^2$, $\Fundefhook[f]{X}{\R^2}$. I punti su cui verificare la proprietà di trasversalità sono l'intersezione, cioé:

$\forall x \in X \cap A$ allora $T_xA + id(T_xX) = \R^2$ \iff $T_xA \oplus T_xX$, infatti, essendo $f$ un'immersione, allora la sua applicazione tangente coincide con l'identità.
%FIGURA di curve trasverse, punti su cui verificare
\end{es}

\begin{defn}[Trasversalità per superfici con bordo]
Se $\partial X \neq \nullset$ definisco $f$ \emph{trasversa} ad $A$ se si ha $f|_{\partial X} \pitchfork A$ (infatti il bordo di una varietà è senza bordo)
%FIGURA?
\end{defn}

\paragraph{Focus}
Si ha allora che:
\begin{itemize}
\item Se $f(X) \cap A = \nullset$ allora $f \pitchfork A$ per vacuità;
\item $\Fundefhook[i]{X}{A \subseteq Y}$ sottovarietà, allora $i \pitchfork Y \iff X \pitchfork Y$, cioè $\forall x \in X\cap Y: T_xX + T_xA = T_xY$.

Inoltre se dim$X +$ dim$A =$ dim$Y \implies T_xX \oplus T_xA = T_xY$ (cioè $T_xX \cap T_xA = \set{0}$);
\item Se $A = \set{y_0}$, $f \pitchfork \set{y_0}$:

$\forall x \in X$ t.c. $f(x) = y_0$ allora $T_{y_0}(A) + D_xf(T_xX) = T_{y_0}Y$ \iff $D_xf$ è surgettiva $\forall x \in f^{-1}(y_0)$, poiché $T_{y_0}(A)$
\end{itemize}

\begin{defn}
Si dice che $y_0 \in Y$ è un \emph{valore regolare} per $f$ se $\forall x \in f^{-1}(y_0)$ si verifica $D_xf$ è surgettiva, mentre $x \in X$ è un \emph{punto critico} per $f$ se $D_xf$ non è surgettiva.
\end{defn}
