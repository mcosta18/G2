%autore: Candido

\begin{oss}

\end{oss}

\titlet{Trasversalità}

\epigraph{Se io vi dico che una cosa è vera, potete anche fregarvene del perché sia vero, insomma... potete anche fregarvene della dimostrazione!\\ In fondo siete fisici... siete abituati a digerire anche di più!}{R.Benedetti}

Si vuole introdurre un concetto secondo cui ci siano alcuni oggetti siano "speciali" rispetto ad altri. Seguono dunque alcuni esempi:

\begin{es}
\begin{itemize}
\item In $\R^2$ quali sono i sottospazi di $dim = 1$ rispetto ad una data retta? Solo se stesso, se come discriminante si considera lo spazio generato dalla retta fissata con l'altra arbitraria. Infatti tutti generano il piano di $dim = 2$, tranne la retta stessa, che con se stessa genera uno spazio di $dim = 1$ (sempre sé);
\item Per i piani nello spazio relativamente a una retta, o le rette relativamente ai piani, o sottospazi in dimensione più elevata si hanno delle estensioni naturali;
\item Per le curve in uno spazio di $dim \geq 2$ si ottiene una nozione analoga considerando i punti di intersezione e gli spazi tangenti in tali punti. Cioè sugli spazi tangenti si hanno le stesse relazioni descritte mediante l'algebra lineare.
%FIGURA con curve intersecanti e spazi tangenti
\end{itemize}
\end{es}

\begin{oss}
La relazione di essere "non speciale" è aperta, poiché per piccole deviazioni da un elemento generico si continuano ad avere elementi generici, inoltre gli elementi generici sono densi nell'insieme di tutti gli elementi.
\end{oss}

Ancora nelle precedente osservazione si stava trattando di concetti intuitivi, si da dunque una definizione formale del concetto di "speciale".

\begin{defn}[Trasversalità]
Sia $Y$ una varietà e $A\subseteq Y$ una sottovarietà, con $\partial Y = \partial A = \nullset$. Sia $X$ compatta e chiusa ($\partial X = \nullset$), e sia $\fundef[f]{X}{Y}$ liscia.
Si definisce allora $f$ \emph{trasversa} ad $A$ se: $\forall x \in X~$t.c.$~y = f(x) \in A~$ si ha che $~T_yY = T_yA + D_xf(T_xX)$
\end{defn}

\begin{es}
%esempio di curve trasverse, punti su cui verificare
\end{es}

\begin{defn}
Se $\partial X \neq \nullset$ definisco $f|_{\partial X} \sout{\cap} A$
\end{defn}
